\chapter{Hey Look at this Stuff!}
\textit{a brief guide to organizing the laboratory store room}\\

The following is advice for PCVs who arrive at a site with a collection of chemicals and/or apparatus for learning science and find that their community is interested in organizing the collection and putting it to use. We strongly recommend that you follow this process with at least one other teacher at your school. We recommend an organizational process with three steps:
\begin{itemize}
\item{getting ready}
\item{the big day}
\item{follow up tasks}
\end{itemize}

After the collection has been organized, there are three more steps:
\item{establishing regular maintenance}
\item{expanding the collection}
\item{using the materials for learning!}

\section{Getting Ready}

\begin{enumerate}
\item{Get permission: before you start a reorganization project, you want to make sure that this is something your school wants to do.}
\item{Involve others: you want to involve other teachers for four reasons. First, it might be their lab. Second, having extra hands is very helpful. Third, they might know what things are that you can't identify or that are no longer labeled. Finally, the whole point of a school getting PCVs is for them to collaborate with other teachers.}
\item{Make time: make an estimate for how long the project will take. Then set aside at least twice that much time. Unless your school has very few things, this project is best done on a weekend or during a holiday when you can take all day if you need.}
\item{Procure materials: get a permanent marker and masking tape or paper and clear packing tape for making labels. Get a washing rag or several that can be permanently dedicated to lab use. Label it as such with the permanent marker. Try to get a rubber blade mop or a similiar tool to let you mop a floor bearing contain chemical residues and glass shards. Get a box of disposable rubber gloves (pharmacy) as well as a bunch of dusts masks if possible. If you do not already have a pair, get some goggles. Get a box of baking soda for neutralizing acid spills. If water is hard to get, plan ahead for having several buckets of water available for this project.}
\item{Ask questions: is your school required to keep broken apparatus for audits by inspectors? Where is a safe place for disposing of broken glass? What is the current protocol for disposing of chemical waste? If there is no procedure, where is the best place for disposing (treated) chemical waste? This is probably a pit latrine, one that will not be pumped for fertilizer.}
\item{Make space: Clean and clear a sorting area as close to the storeroom as possible and as large as possible. A classroom is usually a good choice. Do not use a mess hall / dining area.}
\item{Recruit: If culturally acceptable, retain some truthworthy and reliable students as assistants. Having a small army of hands is extremely helpful for large storeroom projects. Beware that if the store room contains books, in some schools even angelic children will try to steal one or two and this can be a nightmare to undo. You should also watch out for students attempting to perform experiments with the things they discover. While laudable in terms of curousity, for safety reasons you should ask them to hold off until the proper time.}
\item{Read: the list of <extremely-dangerous-chemicals> and <dangerous-chemicals> to be more familiar with what you might encounter. Also read <laboratory-first-aid> and <spill-clean-up> to be prepared.}
\end{enumerate}

\section{The Big Day}
\textit{like a band-aid, better to do it all at once}

Reorganizing a supply room can take anywhere from 5 minutes to 5 days depending on how large a collection is present and what state it is in. This manual recommends doing all of the big work in one go (on one day) and leaving as follow up tasks anything that can be done later. Any number of the follow up tasks can be done on the big day, but there is a set of activities worth doing all at once. The basic idea is to take everything out, sort it, identify future work, clean the storage space, and then put everything back. 

\begin{enumerate}
\item{Eat a big breakfast.}
\item{If you haven't gotten them in advance, fetch several buckets of water. Bring all prepared materials -- permanent pen, tape, googles, gloves, baking soda, etc -- to the work site.}
\item{Move out each item one at a time; use two hands for equipment and carry all glass bottles by the base, not the neck. On the first pass, leave behind anything that is leaking or looks sketchy.}
\item{As you move items out, sort them. First put all apparatus (equipment) in one place and all chemicals in another.
\begin{itemize}
\item{Apparatus: sort the apparatus by type and group similar things together (e.g. all glassware). Make a separate area for broken glassware.}
\item{Chemicals: for rapid sorting, organize the chemicals alphabetically. If you have assistants and many chemicals, you might chalk off a space for each letter. Note that not all letters should be the same size; in English language areas, chemicals starting with C, P, and S tend to be particularly numerous.}
\end{itemize}
}
\item{As chemicals are removed, locate any chemical spills and circle them with chalk. Do not lose track of what compound has been spilled. Label the spill if you know what it was.}
\item{As apparatus are removed, look for shards of broken glass and use paper, cardboard, wood, etc to sweep them into a box or other safe receptical.}
\item{If you remain with some leaking containers of chemicals, do the best you can to transfer the contents of the container into a new clean and dry container. Wear goggles and gloves. Also wear a mask or a cloth over your face to prevent inhaling dust and/or fumes. Label the container before the transfer. If the chemical has a strong smell (e.g. ammonia, glacial acetic acid, diethyl ether), perform the transfer outside where there is better ventillation and stand upwind). If the chemical is on the list of <extremely-dangerous-chemicals> follow the instructions on the list, leave it alone, or ask someone with more experience for advice. If the container is an unlabelled liquid, follow the directions in <identifying-unlabeled-chemicals> before proceeding. This is one reason why this process can take longer than anticipated. Do not pour the contents from a damaged liquid bottle into a partially occupied bottle that claims to be the same thing.}
\item{Clean all spills by following <spill-clean-up>.}
\item{Clean storeroom shelves (if present) with a damp/wet rag that can only be used for this purpose and wear gloves. Mop the floow to flush out remaining debris. Do not dust the shelves or sweet the floor -- the last place you want chemical dust is in the air.}
\item{While the storeroom dries, look through the chemicals. In addition to making sure that they are correctly alphabetized, look for and address the following:
\begin{itemize}
\item{Empty chemical containers. Set these aside to be thoroughly cleaned. Clean them on the big day only if you have extra time/labor or if you need them for storing another chemical.}
\item{Cracked caps or otherwise damaged containers. If possible, , bottles with  - deposition (often white) on tops of bottles --> check the container for cracks, use a rag in a gloved hand to wipe up deposition
\item{For liquid chemicals needing a new home, be wary of pouring a chemical into a container of another chemical with the same name. If one bottle is mislabeled (perhaps it was refilled with something else), the result could be catastrophic. If you feel that you must combine reagents, first use the instructions for <identifying-unknown-chemicals> to confirm that the chemicals are indeed the same and then try mixing a small amount from each bottle in another container to confirm that nothing violent occurs. Then, if you do add one to the other, add it very slowly. But again, the best plan is to find a new container, ideally of the same material as the original container.}
\item{Containers of solid reagents that sounds like they are full of water (they slosh). See <salvaging-deliquescent-salts> for more advice.}
\item{Any of the chemicals on the list of <extremely-dangerous-chemicals>. Move these chemicals away from the others. They should not be filed in the main collection for regular use, but placed somewhere secure where there whereabouts is known but where they will not be used. This could be a specially labeled corner of the storeroom.}
\item{Fading or peeling labels. Make a new label right away. Be sure to include the chemical name and exact formula. For bottles of acids, also include the specific gravity (SG) or density (a number between 1 and 2), the \% purity, and the molecular mass.}
\item{Unlabled containers. Group these in a special place for a follow-up project. If you have a hunch based on where the chemical was found, make a lable that shares both your guess and your doubt, e.g. ``unknown, may be copper (II) chloride.''}
\end{itemize}
}
\item{}
\end{enumerate}

\section{Follow Up Tasks}

NaOH / KOH --> dilute ~10x, test with titration
	- FeCl3 --> dilute ~2x, transfer to a suitable container
 - empty containers --> wash and reuse
